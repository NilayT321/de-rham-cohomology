\documentclass{beamer}
\usetheme{Madrid}
\usepackage{format}
\usepackage{helvet}

\title[de Rham Cohomology]{de Rham Cohomology \& Stokes' Theorem}
\subtitle{Math 412 -- Final Project}
\author{Nilay Tripathi}
\date{May 3rd, 2023}

\svgpath{figs/}

\begin{document}
		\maketitle

		\begin{frame}{Overview}
				Differential forms provide a unifed approach to integrate in higher dimensions (and also over arbitrary manifolds). 
				\only<2->{
						\begin{columns}
								\column{0.4\textwidth} 
								We have \alert{Stokes' theorem}:
								\begin{equation*}
										\int_{\prtl E} \omega = \int_E d\omega 
								\end{equation*}

								\column{0.6\textwidth}
										\begin{figure}[H]
												\centering
												\includesvg[width = 0.7\textwidth]{manifoldE}
										\end{figure}
						\end{columns}
				}
				\only<3->{But what happens when Stokes' theorem fails, or when it cannot be used?}
		\end{frame}

		\begin{frame}{Overview \& Idea}
				\begin{block}{Idea}
						Stokes' theorem fails in the presence of \emph{holes}. The \alert{de Rham cohomology} is a tool to measure how many holes a space has.
				\end{block}
				We will: 
				\begin{itemize}
						\item<2-> Review some basic terminology to be used in the discussion of Stokes' theorem 
						\item<3-> A brief introduction to the idea of homology, which can be used to detect holes. 
						\item<4-> Introducing the de Rham cohomology as a tool to measure holes 
						\item<5-> Applications (\alert{work on these})
				\end{itemize}
		\end{frame}

		\begin{frame}{Cells}
				\begin{definition}[$k$-Cell]
						A \alert{$k$-cell} on a space $U \subseteq \R^n$ is a continuous map from $\alpha: [0,1]^k \to U$. 
				\end{definition}
				\begin{figure}[H]
						\centering
						\includesvg[width = 0.4\textwidth]{cells}
						\caption{Some cells: A 0-cell $\sigma^0$, 1-cell $\sigma^1$, and a 2-cell $\sigma^2$}.
				\end{figure}
		\end{frame}
\end{document}
