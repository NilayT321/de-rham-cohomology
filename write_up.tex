\documentclass{article}
\usepackage{format}

\title{Stokes' Theorem \& The de Rham Cohomology}
\author{Nilay Tripathi}
\date{May 3, 2024}

\svgpath{figs/}

\begin{document}
		\maketitle

		\tableofcontents
		
		\section{Overview}
		For this presentation, we will first review differential forms, particularly in two- and three-dimensions as well as the concepts of exterior differentiation and the integration of differential forms. From this point, we will then briefly review Stokes' theorem, which serves as a generalization to the fundamental theorem of calculus as well as the Green's theorem, Stokes' theorem, and divergence theorem studied in multivariable calculus. This will require properties of closed and exact forms, which we also discuss. \par 

		The main focus of the presentation is to analyze cases where the Stokes' theorem fails in higher dimensional settings and spaces. The failure of the Stokes' theorem is caused by the presence of ``holes'' in spaces, which can be detected using the theories of homology and cohomology. We first present an overview of using homology to detect holes and give a geometric defintion of the homology groups. We then develop the de Rham cohomology, which uses differential forms and the exterior derivative to detect holes in spaces. It will turn out that the computed de Rham cohomology groups will effectively ``measure'' the amount of holes in the space, and reveals the extent to which the Stokes' theorem fails in arbitrary spaces. 

		\section{Overview of Differential Forms}
		We first review the concept of differential forms. The high-level/geometric approach I use here is based off that in Evan Chen's resource \emph{The Napkin} \cite{napkin}, namely chapters 44 \& 45, which emphasizes a more geometric/visual understanding of the concepts rather than formal proofs. More formal definitions are gleaned using the course notes as well as Spivak's \emph{Calculus on Manifolds} \cite{spivak} and Munkres' \emph{Analysis on Manifolds} \cite{munkres}.

		\subsection{0-Forms, 1-Forms, \& 2-Forms} 
		According to \cite{spivak}, a 0-form is simply just a differentiable function $f: U \to \R$, where $U \subseteq \R^n$ is open. In our presentation, we will mostly stick to the case when $n = 2$ or $3$. 
		\begin{defn}[Differential Form]
				We say that $\omega$ is a \textbf{differential $k$-form} provided that $\omega \in \Lambda^k (\R^n_p)$. Recall that the latter is the set of alternating tensors on the tangent space of a point $p\in\R^n$. 
		\end{defn}

        We will not focus extensively on the formal definitions, but rather on the main concepts/computational aspects. A \textbf{1-form} on $\R^n$ can be written as 
        \begin{equation*}
            \omega = f_1(p)\ dx_1 + f_2(p)\ dx_2 + \cdots + f_n(p)\ dx_n
        \end{equation*}
        Where the $f_i: \R^n \to \R$ are all differentiable functions. So for $\R^3$, a typical 1-form looks like 
        \begin{equation*}
            \omega = f_1(p)\ dx + f_2(p)\ dy + f_3(p)\ dz 
        \end{equation*}
        A \textbf{2-form} on $\R^3$ is something of the form 
        \begin{equation*}
            \omega = f_1(p)\ dx \wedge dy + f_2(p)\ dy \wedge dz + f_3(p)\ dx \wedge dz 
        \end{equation*}
        In general, recall that differential forms are just tensors, so we only need to specify the coefficients of the basis vectors. For $\R^n$, which has dimension $n$, there are $\binom{n}{k}$ basis vectors of $\Lambda^k(\R^n)$, so these many terms are needed to specify a $k$-form on $\R^n$.

        \subsection{Exterior Derivative}
        If we have a differential $k$-form, how do we obtain a $(k+1)$-form from it? This is what the exterior derivative/differential does. 
        \begin{defn}[Exterior Differential]
            Given a differential 0-form, just a function $f: U\to \R$, the \textbf{differential} of $f$ is 
            \begin{equation*}
                df = \sum_{i=1}^n \frac{\partial f}{\partial x_i}dx_i
            \end{equation*}
            Given a differential 1-form $\omega = \sum_{i=1}^n f_i\ dx_i$, its exterior derivative is given by 
            \begin{align*}
                d\omega = d\left( \sum_{i=1}^n \frac{\partial f}{\partial x_i}\ dx_i \right) = \sum_{i=1}^n \sum_{j=1}^n \frac{\partial^2 f}{\prtl x_j \prtl x_i}\ dx_j \wedge dx_i
            \end{align*}
        \end{defn}
        Notice that if $f$ has twice-continuous partial derivatives, then $d^2 \omega = 0$, since the partials are symmetric while the wedge product is anti-symmetric. In fact, this holds in general for any $k$-form. 
        \begin{prop}
            For any differential $k$-form $\omega$, we have that $d^2\omega = 0$.
        \end{prop}
        Let's compute some concrete examples. Suppose that we have a 1-form in $\R^3$ given as 
        \begin{equation*}
            \omega = P\ dx + Q\ dy + R\ dz
        \end{equation*}
        Then we see that 
        \begin{align*}
            d\omega &= \left( \frac{\prtl P}{\prtl x}\ dx + \frac{\prtl Q}{\prtl y}\ dy + \frac{\prtl R}{\prtl z}\ dz \right) \wedge dx \\ 
            &+ \left( \frac{\prtl P}{\prtl x}\ dx + \frac{\prtl Q}{\prtl y}\ dy + \frac{\prtl R}{\prtl z}\ dz \right) \wedge dy \\ 
            &+ \left( \frac{\prtl P}{\prtl x}\ dx + \frac{\prtl Q}{\prtl y}\ dy + \frac{\prtl R}{\prtl z}\ dz \right) \wedge dz
        \end{align*}
        Notice that the terms $dx \wedge dx$, $dy \wedge dy$, and $dz\wedge dz$ all are 0. In addition, by the anti-symmetry of the wedge product, we see that 
        \begin{equation*}
            d\omega = \left( \frac{\prtl P}{\prtl x} - \frac{\prtl Q}{\prtl y} \right)\ dx \wedge dy + \left( \frac{\prtl Q}{\prtl y} - \frac{\prtl R}{\prtl z} \right)\ dy \wedge dz + \left( \frac{\prtl P}{\prtl x} - \frac{\prtl R}{\prtl z} \right)\ dx \wedge dz
        \end{equation*}
        Notice that $d\omega$ is indeed a two-form. Also observe that the terms in the parentheses are just the curl. Indeed, the exterior derivative is the tool which will help generalize the notions of curl, divergence, etc. 

        \subsection{Integration of Forms \& Stokes' Theorem}
        A differential $k$-form's life purposes is to be integrated over some $k$-dimensional \emph{cell}. A $k$-cell can be seen as a smooth map from $\alpha: [0,1]^k \to X$. Once can embed these cells into a particular space to visualize them, as seen in \cref{fig:cells}.
        \begin{figure}
            \centering
            \includesvg[width = 0.5\textwidth]{cells}
            \caption{Some cells: $\sigma^0$ a 0-cell, $\sigma^1$ a 1-cell, and $\sigma^2$ a 2-cell.}
            \label{fig:cells}
        \end{figure}
		\bibliographystyle{plain}
		\bibliography{refs}
\end{document}
